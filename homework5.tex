\documentclass{article}
\usepackage{listings}
\usepackage{amsmath}
\usepackage{fullpage}
\usepackage{tabularx}
\usepackage{graphicx}
\usepackage{cite}
\begin{document}
\lstset{language=Java}
\title{CS350 homework 5}
\author{Yuchen Hou}
\maketitle

\section{LCS with constrains}
Assume the size of the alphabet and the length of $\alpha, \beta$ are $v_0, v_1,
v_2$. The following algorithm can find $\gamma$. In this problem an FA is viewed
as a graph and M(x, y) is the notation for FA M with x nodes and y edges.
\begin{enumerate}
  \item construct FA $M_1(O(v_1), O(v_1))$ accepting Reg $L_1$ = \{subsequences
  of $\alpha$\};
  \item construct FA $M_2(O(v_2), O(v_2))$ accepting Reg $L_1$ = \{subsequences
  of \ $\beta$\};
  \item construct FA $M_3(O(v_0), O(v_0))$ accepting Reg $L_3$ = \{strings not
  containing substring abb\};
  \item construct FA $M_5(O(v_0v_1v_2), O(v_0^2v_1^2v_2^2)) =
  M_1 \times M_2 \times M_4$ accepting Reg $L_5 = L_1 \cap L_2 \cap L_3$ using
  Cartesian product and perform $\lambda$ elimination ($M_1, M_2$ accept
  only finite strings, so $M_5$ accpets only finite strings and is a DAG);
  \item return the longest string $M_5$ accepts as $\gamma$ using shortest
  path algorithm with edge weight negation;
\end{enumerate}
The time complexity is the sum of all above steps: $O(v_0^2v_1^2v_2^2)$.

\section{LCS for Reg}
The following algorithm can compute the number D. In this problem an FA is
veiwed as a graph.
\begin{enumerate}
  \item construct FA $M_1$ accepting Reg $L_1'$ = \{subsequences of strings in
  $L_1$\};
  \item construct FA $M_2$ accepting Reg $L_2'$ = \{subsequences of strings in
  $L_2$\};
  \item construct FA $M_3  = M_1 \times M_2$ accepting Reg $L_3 = L_1' \cap
  L_2'$ using Cartesian product and perform $\lambda$ elimination;
  \item partitian $M_3$ into SCCs;
  \item for every SCC, delete it if the initial state can't reach it or it can't
  reach any final state;
  \item if any remaining SCC is not a singleton, return infinity as D; otherwise
  $M_3$ is a DAG, return the longest string $M_3$ accepts as D using shortest path
  algorithm with edge weight negation;
\end{enumerate}

\section{Longest common subsequence for quasi strings}
Every point in a circle can be represented as a 3-tuple: (radius, color, angle).
The following algorithm returns true if the 2 circles are similar and returns
false if they are not.
\begin{enumerate}
  \item sort all points in $C_1$ by radius, then by color(any ordering for
  color works), then by angle, and copy the sorted points into an array $S_1$
  and conceal their angle attributes from following operations(3-tuples become
  2-tuples); do the same for $C_2$ (into an array $S_2$);
  \item regard the 2 arrays as 2 strings and every point in them as a
  character(regard 2 points as same characters if their 2-tuple representation
  are the same);
  \item construct FA $M_1$ accepting Reg $L_1$ = \{subsequences of $S_1$\};
  \item construct FA $M_2$ accpeting Reg $L_2$ = \{subsequences of $S_2$\};
  \item construct FA $M_3  = M_1 \cap M_2$ accepting Reg $L_3 = L_1 \cap L_2$
  using Cartesian product ($M_1, M_2$ accept only finite strings, so $M_3$
  accpets only finite strings and is a DAG);
  \item find all strings longer than k $M_3$ accepts using DFS for DAG
  and put them into a set L;
  \item loop through every string s in L, do this: (reveal the angle
  attribute pair(one for the point in C1, one for the point in C2) for every character c in s, calculate
  the diffrences of the pair and mod them by $2 \pi$; if all these differences
  are the same, return true); if the loop finishes without returning true,
  return false
\end{enumerate}
\end{document}
