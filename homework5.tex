\documentclass{article}
\usepackage{listings}
\usepackage{amsmath}
\usepackage{fullpage}
\usepackage{tabularx}
\usepackage{graphicx}
\usepackage{cite}
\begin{document}
\lstset{language=Java}
\title{CS350 homework 5}
\author{Yuchen Hou}
\maketitle
\section{Longest common subsequence for strings with constrains}
Assume the length of $\alpha, \beta$ are $v_1, v_2$. Assume the size of
the alphabet is $v_0$. The following algorithm with (time complexity noted) can
find $\gamma$. Note that in this problem an FA is viewed as a graph and M(x, y)
is the notation for FA M with x nodes and y edges.
\begin{description}
  \item[$O(v_1^2)$] construct FA $M_1(O(v_1), O(v_1))$ to accept all
  subsequences of $\alpha$ using skipping method
  \item[$O(v_2^2)$] construct FA $M_2(O(v_2), O(v_2))$ to accept all
  subsequences of $\beta$ using skipping method
  \item[$O(v_0)$] construct FA $M_3(O(v_0), O(v_0))$ to accept all strings
  containing substring abb using Kleene theorem
  \item[$O(v_0)$] construct FA $M_4(O(v_0), O(v_0)) = \overline{M_3}$ using
  the method formulated in cs317 homework3 problem3
  \item[$O(v_1v_2v_0)$] construct FA $M_5(O(v_0v_1v_2), O(v_1 + v_2 +
  v_0)) = M_1 \cap M_2 \cap M_4$ using Cartesian product method ($M_1, M_2$
  accept only finite strings, so $M_5$ accpets only finite strings, so it's a
  DAG)
  \item[$O(v_1^2v_2^2v_0)$] find the longest acceptable string for $M_5$ and
  return it as $\gamma$ using longest path algorithm for DAG
\end{description}
The time complexity is the sum of all above steps: $O(v_1^2v_2^2v_0)$.
\section{Longest common subsequence for languages}
The following algorithm can compute the number D. In this problem, an FA is
veiwed as a graph.
\begin{enumerate}
  \item construct FA $M_1$ to accept all subsequences of strings in $L_1$ using
  skipping method discussed in class
  \item construct FA $M_2$ to accept all subsequences of strings in $L_2$ using
  skipping method discussed in class
  \item construct FA $M_3  = M_1 \cap M_2$ using Cartesian product method and
  perform $\lambda$ elimination
  \item partitian $M_3$ into SCCs
  \item for every SCC, delete it if the initial state can't reach it or it can't
  reach any final state
  \item if any remaining SCC has any loop, return infinity as D; otherwise $M_3$
  is a DAG, return the longest acceptable string length for $M_3$ as D using
  longest path algorithm for DAG
\end{enumerate}
\section{Longest common subsequence for quasi strings}
Every point in a circle can be represented as a 3-tuple: (radius, color, angle).
The following algorithm returns true if the 2 circles are similar and returns
false if they are not.
\begin{enumerate}
  \item sort all points in $C_1$ by radius, then by color(any ordering for
  color works), then by angle, and copy the sorted points into an array $S_1$
  and conceal their angle attributes from following operations(3-tuples become
  2-tuples); do the same for $C_2$ (into an array $S_2$)
  \item regard the 2 arrays as 2 strings and every point in them as a
  character(regard 2 points as same characters if their 2-tuple representation
  are the same)
  \item construct FA $M_1$ that accpets all subsequences of string $S_1$
  \item construct FA $M_2$ that accpets all subsequences of string $S_2$
  \item construct FA $M_3  = M_1 \cap M_2$ using Cartesian product method
  \item $M_1, M_2$ accept only finite strings, so $M_3$ accpets only finite
  strings, so it's a DAG
  \item find all acceptable strings longer than k for $M_3$ using DFS for DAG
  and put all strings into a set L
  \item loop through every string s in L, do this: (reveal the angle
  attribute pair(one for the point in C1, one for the point in C2) for every character c in s, calculate
  the diffrences of the pair and mod them by $2 \pi$; if all these differences
  are the same, return true); if the loop finishes without returning true,
  return false
\end{enumerate}
\end{document}
