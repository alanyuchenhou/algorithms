\documentclass{article}
\usepackage{listings}
\usepackage{amsmath}
\usepackage{fullpage}
\usepackage{tabularx}
\usepackage{graphicx}
\usepackage{cite}
\begin{document}
\lstset{language=Java, tabsize=4}
\title{CS350 homework 6 generalization}
\author{Yuchen Hou}
\maketitle
\section{Problem}
This is a generalization of homework 6 problem 2. Suppose G(V, E) is a decorated
graph. Each edge in G has a color attribute. The color attribute has m posible
values: $color_1, color_2, color_3 \ldots color_m$. Design a algorithm to
answer this question: is it true that, for every path p from v to v', the
following requirement is satisfied:
\begin{align*}
f(p) = k_1c_1 + k_2c_2 + k_3c_3 \ldots k_nc_n > k_0
\end{align*}
where $n < m$, and k's are constants and $c_i$ is the number of edges with
$color_i$ in the path.
\section{Algorithm}
This problem can be solved in 2 steps. The steps and the time complexity of each
step are as follows:
\begin{description}
  \item[O(e)] assign each edge a weight attribute: edges with $color_i$ have
  weight $k_i$; edges with colors that do not exist in the requiremnt have
  weight 0 (we don't care about these colors);
  \item[O(ve)] compute the length of the shortest path from v to v'; if the
  length is greater than $k_0$, the answer is yes; otherwise, the answer is no;
\end{description}
The total time complexity is O(ve).
\section{Proof}
After assigning a weight attribute to each edge, the
length of a path p is
\begin{align*}
length(p) = k_1c_1 + k_2c_2 + k_3c_3 ... k_nc_n = f(p)
\end{align*}
So the original question is equivalent to: is it ture that, $\forall$ path p
from v to v' $length(p) > k_0$, which is equivalent to: is the length
of the shortest path from v to v' greater than $k_0$, which can then be
efficiently solved using shortest path algorithm.
\section{Handle negative cycles}
If there are negative cycles on the path, G will not have a shortest path. In
this case, the correct answer is no and this algorithm will still give the
correct answer as long as the shortest path algorithm it uses can detect negative cycles (e.g.
Bellman-Ford shortest paths algorithm). Specifically, the shortest path
algorithm can indicate the existence of a negative cycle by either raising an
exception or return $-\infty$. In either way, the condition - the shortest
path length greater than $k_0$ - will not be met and the algorithm answers no.
\end{document}
