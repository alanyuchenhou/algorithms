\documentclass{article}
\usepackage{listings}
\usepackage{amsmath}
\usepackage{fullpage}
\usepackage{tabularx}
\usepackage{graphicx}
\usepackage{cite}
\begin{document}
\lstset{language=Java, tabsize=4}
\title{CS350 homework 4}
\author{Yuchen Hou}
\maketitle

\section{Bad closest pair}
\subsection{Recursive relation}
Assume $A_1$ has r points. So $A_2$ has n-r points. Then
\begin{align*}
T_{bad}(n) &= \sum_r P(r) (T_{bad}(r) + T_{bad}(n-r) + T_{boundary}(r, n))\\
T_{split}(n) &= O(n)\\
T_{boundary}(r, n) &= O(r(n-r))
\end{align*}
\subsection{Probability}
r follows binomial distribution:
\begin{align*}
r \sim B(n, p)
\end{align*}
and its probability mass function is:
\begin{align*}
f(r, n, p)
&= \binom{n}{r}p^r(1-p)^{(n-r)}\\
&=\frac{n!}{r!(n-r)!}(\frac{1}{2})^n\\
\end{align*}
When n is large enough, a reasonable approximation to B(n, p) is normal
distribution:
\begin{align*}
r &\sim N(\mu, \sigma^2)\\
\mu &= np = \frac{n}{2} \\
\sigma^2 &= np(1-p) = \frac{n}{4}\\
\sigma &= \frac{\sqrt{n}}{2}
\end{align*}
and its probability density function is:
\begin{align*}
f(r, \mu, \sigma)
&= \frac{1}{\sigma \sqrt{2\pi}}\exp(-\frac{(r-\mu)^2}{2\sigma^2})\\
f(r, n, p)
&= \frac{1}{\frac{\sqrt{n}}{2}
\sqrt{2\pi}}\exp(-\frac{(r-\frac{n}{2})^2}{2(\frac{\sqrt{n}}{2})^2})\\
&=\frac{\sqrt{2}}{\sqrt{n\pi}}\exp(-2\frac{(r-\frac{n}{2})^2}{n}) = P(r)
\end{align*}
\subsection{Statement}
$T_{bad}(n) = O(n^2 \log n)$
\subsection{Proof by induction}
Induction hypothesis: $\forall m < n, T_{bad}(m) = O(m^2 \log m) < am^2 \log m$
\begin{align*}
T_{bad}(n)
&= T_{split}(n) + \sum_r P(r) (T_{bad}(r) + T_{bad}(n-r) + T_{boundary}(r, n))\\
&= O(n) + \sum_r f(r, n, p) (O(r^2 \log r) + O((n-r)^2 \log (n-r)) +
O(r(n-r)))\\
&= O(n) + \sum_r \frac{\sqrt{2}}{\sqrt{n\pi}}
\exp(-2\frac{(r-\frac{n}{2})^2}{n})(ar^2 \log r + a(n-r)^2 \log (n-r) + cr(n-r))\\
&= O(n) + \frac{\sqrt{2}}{\sqrt{n\pi}} \int_r
\exp(-2\frac{(r-\frac{n}{2})^2}{n})(ar^2 \log r + a(n-r)^2 \log (n-r) + cr(n-r)) dr\\
&\ldots\\
&< an^2 \log n \text{ // choose a large enough a} \\
\implies T_{bad} &= O(n^2 \log n)
\end{align*}

\section{Naive Karatsuba algorithm}
Let a = 1.59.
\begin{align*}
T_{naive}(n)
&= \sum_i T_k(\vert \beta_i \vert)\\
&= \sum_i T_k(in)\\
&= \sum_i O((in)^{a})\\
&= n^{a} \sum_i  O(i^{a})\\
&= n^{a} \int_i  O(i^{a}) di\\
&= n^{a} O(i^{a + 1})\\
&= O(n^{4.18})
\end{align*}

\section{Better Karatsuba algorithm}
Let a = 1.59.
This recursive algorithm essentially applys Karatsuma algorithm to pairs
of string with length n, 2n, 4n, 8n ... And its total time cost is the sum
of all the time costs of Karatsuma multiplications.
\begin{align*}
T_{better}(n)&= \frac{n}{2}T_k(n) \text{ // $\frac{n}{2}$ multiplications for
strings with length n} \\
&+ \frac{n}{4}T_k(2n) \text{ // $\frac{n}{4}$ multiplications for strings with
length 2n} \\
&+ \frac{n}{8}T_k(4n) \text{ // $\frac{n}{8}$ multiplications for strings with
length 4n} \\
& \ldots \\
&+ \frac{n}{2^{i}}T_k(2^{i-1}n) \text{ // $\frac{n}{2^{i}}$ multiplications
for strings with length $2^{i-1}n$} \\
& \ldots \\
&+ \frac{n}{n}T_k(\frac{n}{2}n) \text{ // $1$ multiplications for strings with
length $\frac{n}{2}n$} \\
&= \sum_{i = 1}^{\log n} \frac{n}{2^{i}} T_k(2^{i-1}n) \\
&= \sum_{i = 1}^{\log n} \frac{n}{2^{i}} O((2^{i-1}n)^a) \\
&= O(n^{a+1}) \sum_{i = 1}^{\log n} \frac{1}{2^{i}} (2^{i-1})^a \\
&= O(n^{a+1}) \sum_{i = 1}^{\log n} (2^{a-1})^{i} \\
&= O(n^{a+1}) (2^{a-1})^{\log n} \\
&= O(n^{a+1}) n^{a-1} \\
&= O(n^{2a}) \\
&= O(n^{3.18})
\end{align*}

\section{Closest airplanes}
\subsection{Formulation}
This is a closest pair problem. $\forall p_1, p_2 \in planes \ d(p_1, p_2) \leq
n \implies$ all p's are in a same cube of dimension (n, n, n). We can devide
this cube into $n^3$ unit cubes with dimension (1, 1, 1). $d \geq 1 \implies$
each cube can contain at most 8
points. $\#_{point} = \#_{cube} \implies \forall cube$ cube contains 1
point $\lor \exists cube$ cube contains more than 1 points. Eitherway, the
closest pair must be in 1 of 27 adjacent cubes(the pair can be in the same cube).
\subsection{Algorithm}
\begin{lstlisting}
(Point, Point) findClosestPair(ArrayList<Point> points) {
	Float shortestDistance = n;
	Point p1 = NULL;
	Point p2 = NULL;
	for (Point point1: points) {
		for (Point point2: points.getPointsInAdjacentCubesOf(point1) {
			Float newDistance = distance(point1, point2);
			if (newDistance < shortestDistance) {
				shortestDistance = newDistance;
				p1 = point1;
				p2 = point2;
			}
		}
	}
	return (p1, p2);
}
\end{lstlisting}
\subsection{Time complexity}
The outer for loop has $n^3$ iterations and the inner for loop has constant
number of iterations (8 points per cube, 27 adjacent cubes). So the time
complexity is:
\begin{align*}
T_{close}(n^3) = n^3 O(1) = O(n^3)
\end{align*}

\section{Closest strings}
\subsection{Formulation}
We can reduce the problem to a closest pair problem: convert every string S to a
point (x, y) and return the shortest distance of all points:
\begin{align*}
x =& \#_a(S) \in \{0, 1, 2, \ldots m\}\\
y =& \#_b(S) \in \{0, 1, 2, \ldots m\}\\
distance(S_1, S_2) =& (x_1-x_2)^2 + (y_1-y_2)^2
\end{align*}
Note there are only $(m + 1)^2$ possible distinct points. So $n > (m + 1)^2
\implies \exists p_1, p_2 \in points: distance(p_1, p_2) = 0$.
\subsection{Algorithm}
\begin{lstlisting}
Integer findSmallestDistance(ArrayList<String> strings) {
	ArrayList<Point> points;
	for (String string: strings) {
		points.add(Point(#_a(string), #_b(string)));
	}
	if (n > (m + 1)^2) {
		return 0;
	} else {
		return findClosestPair(points);
	}
}
\end{lstlisting}
\subsection{Time complexity}
The for loop has time complexity nO(m) = O(mn). If $n > (m + 1)^2$:
\begin{align*}
T_{close}(n)
=& O(mn) + O(1)\\
=& O(mn)
\end{align*}
otherwise $n \leq (m + 1)^2 \implies O(\log(n) < O(\log(m)) < O(m))$:
\begin{align*}
T_{close}(n)
=& O(nm) + O(n \log n) \\
=& O(nm)
\end{align*}
\end{document}
