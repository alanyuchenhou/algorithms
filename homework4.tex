\documentclass{article}
\usepackage{listings}
\usepackage{amsmath}
\usepackage{fullpage}
\usepackage{tabularx}
\usepackage{graphicx}
\usepackage{cite}
\begin{document}
\lstset{language=Java}
\title{CS350 homework 4}
\author{Yuchen Hou}
\maketitle

\section{Bad closest pair}
\subsection{Recursive relation}
Assume $A_1$ has r points. So $A_2$ has n-r points. Then
\begin{align*}
T_{bad}(n,r) &= T_{split}(n) + T_{bad}(r) + T_{bad}(n-r) + T_{boundary}(r, n)\\
T_{split}(n) &= O(n)\\
T_{boundary}(n) &= O(r(n-r))
\end{align*}
\subsection{Statement}
$T_{bad}(n) = O(n^2)$
\subsection{Proof by induction}
r follows binomial distribution:
\begin{align*}
r \sim B(n, p)
\end{align*}
and its probability mass function is:
\begin{align*}
f(r, n, p)
&= \binom{n}{r}p^r(1-p)^{(n-r)}\\
&=\frac{n!}{r!(n-r)!}(\frac{1}{2})^n\\
\end{align*}
When n is large enough, a reasonable approximation to B(n, p) is normal
distribution:
\begin{align*}
r &\sim N(\mu, \sigma^2)\\
\mu &= np = \frac{n}{2} \\
\sigma^2 &= np(1-p) = \frac{n}{4}\\
\sigma &= \frac{\sqrt{n}}{2}
\end{align*}
and its probability density function is:
\begin{align*}
f(r, \mu, \sigma)
&= \frac{1}{\sigma \sqrt{2\pi}}\exp(-\frac{(r-\mu)^2}{2\sigma^2})\\
f(r, n, p)
&= \frac{1}{\frac{\sqrt{n}}{2}
\sqrt{2\pi}}\exp(-\frac{(r-\frac{n}{2})^2}{2(\frac{\sqrt{n}}{2})^2})\\
&=\frac{\sqrt{2}}{\sqrt{n\pi}}\exp(-2\frac{(r-\frac{n}{2})^2}{n})\\
\end{align*}
Induction hypothesis: $\forall m < n, T_{bad}(m) =
O(m^2) < am^2$
\begin{align*}
T_{bad}(n)
&= \sum_r P(r, n) T_{bad}(n,r)\\
&= \sum_r f(r, n, p) (O(n) + O(r^2) + O((n-r)^2) + O(r(n-r)))\\
&= \sum_r \frac{\sqrt{2}}{\sqrt{n\pi}}\exp(-2\frac{(r-\frac{n}{2})^2}{n})(bn +
ar^2 + a(n-r)^2 + cr(n-r))\\
&= \frac{\sqrt{2}}{\sqrt{n\pi}} \int_r \exp(-2\frac{(r-\frac{n}{2})^2}{n})(bn +
ar^2 + a(n-r)^2 + cr(n-r)) dr\\
&< \sqrt{\frac{2}{\pi}}(\frac{2}{3}an^2 + \frac{1}{6}cn^2 + bn)\\
&< an^2 \text{ // choose a large enough a} \\
\implies T_{bad} &= O(n^2)
\end{align*}
\section{Naive Karatsuba algorithm}
In the worst case, for the ith multiplication:
\begin{align*}
\vert \beta_i \vert = in
\end{align*}
and the time complexity for each multiplication uising Karatsuba algorithm is:
\begin{align*}
T_k(\vert \beta_i \vert) = O(\vert \beta_i \vert^{1.59}) < a (in)^{1.59}
\end{align*}
And the total time complexity is:
\begin{align*}
T_{naive}(n)
&= \sum_i T_k(\vert \beta_i \vert)\\
&< \sum_i a (in)^{1.59}\\
&= a n^{1.59} \sum_i  i^{1.59}\\
&= a n^{1.59} \int_i  i^{1.59} di\\
&= a n^{1.59} b n^{2.59} di\\
&= c n^{4.18}\\
\implies T_{naive}(n) &= O(n^{4.18})
\end{align*}
\section{Better Karatsuba algorithm}
Let a = 1.59. From Karatsuma algorithm:
\begin{align*}
T_k(n) = O(n^a) < bn^a
\end{align*}
This recursive algorithm essentially applys Karatsuma algorithm to pairs
of string with length n, 2n, 4n, 8n ... nn. And its total time cost is the sum
of all the time costs of Karatsuma multiplications.
\begin{align*}
T_{better}(n)&= \frac{n}{2}T_k(n) \text{ // $\frac{n}{2}$ multiplications for
strings with length n} \\
&+ \frac{n}{4}T_k(2n) \text{ // $\frac{n}{4}$ multiplications for strings with
length 2n} \\
&+ \frac{n}{8}T_k(4n) \text{ // $\frac{n}{8}$ multiplications for strings with
length 4n} \\
& \ldots \\
&+ \frac{n}{2^{i}}T_k(2^{i-1}n) \text{ // $\frac{n}{2^{i}}$ multiplications
for strings with length $2^{i-1}n$} \\
& \ldots \\
&+ \frac{n}{n}T_k(2^{\log n -1}n) \text{ // $1$ multiplications for
strings with length $2^{\log n -1}n$} \\
&= \sum_{i = 1}^{\log n} \frac{n}{2^{i}} T_k(2^{i-1}n) \\
&< \sum_{i = 1}^{\log n} \frac{n}{2^{i}} b(2^{i-1}n)^a \\
&< bn^{a+1} \sum_{i = 1}^{\log n} \frac{1}{2^{i}} (2^{i-1})^a \\
&< \frac{bn^{a+1}}{2} \sum_{i = 1}^{\log n} \frac{1}{2^{i-1}} (2^a)^{i-1} \\
&< \frac{bn^{a+1}}{2} \sum_{i = 0}^{\log n - 1} \frac{1}{2^i} (2^a)^i \\
&< \frac{bn^{a+1}}{2} \sum_{i = 0}^{\log n - 1} (2^{a-1})^i \\
&= \frac{bn^{a+1}}{2} \frac{1- z^{\log n}}{1 - z}
\end{align*}
where $z = 2^{a-1} = 2^{0.59}$. Thus
\begin{align*}
T_{better}(n) = c n^{2.59} n^{0.59} = O(n^{3.18})
\end{align*}
\section{Closest airplanes}
\subsection{Senario descriptions}
Because $d \leq n$, all points(airplanes) are within a ball of diameter n and
center P. This ball is bounded by a cube of dimension (n, n, n) and center P.
The volume of the cube is
\begin{align*}
V = n^3
\end{align*}
This cube can be divided into $n^3$ unit cubes with dimension (1, 1, 1). Because
$d \geq 1$, each cube can contain at most 8 points. Assume the closest pair has
disdance x. As there are same amount of points and cubes, either every cube has
exactely 1 point or one cube has more than 1 points. Eitherway, the closest
pair must be in the same cube or adjacent cubes.
\subsection{Algorithm}
\begin{lstlisting}
(Point, Point) findClosestPair(ArrayList<Point> points) {
	closestDistance = 1000n;
	Point p1 = NULL;
	Point p2 = NULL;
	for (Point point1: points) {
		for (Point point2: point1.findAllPointsInAdjacentCubes() {
			if (distance(point1, point2) < closestDistance) {
				p1 = point1;
				p2 = point2;
				closestDistance = distance(point1, point2);
			}
		}
	}
	return (p1, p2);
}
\end{lstlisting}
\subsection{Time complexity}
The outer for loop has $n^3$ iterations and the inner for loop has at most $8
\times 27$ iterations (8 points per cube, 27 adjacent cubes, at maximum).
All the processing within the inner loop and function
findAllPointsInAdjacentCubes() cost time O(1). So the total time complexity is
\begin{align*}
T_{close}(n^3) = n^3 8 \times 27 O(1) = O(n^3)
\end{align*}
\section{Closest strings}
This problem can be formulated as a closest points problem. Every string S can
be represented as a point P(x, y):
\begin{align*}
x =& \#_a(S)\\
y =& \#_b(S)\\
x <& m // because \vert S \vert < m \\
y <& m // because \vert S \vert < m \\
d(S_1, S_2) =& (x_1-x_2)^2 + (y_1-y_2)^2 \geq 1
\end{align*}
Therefore, all points are in a square of dimension $(m, m)$. Notice that for the
preprocessing, the conversion of a string to a point has time complexity
O(m), and the conversion of all strings has time complexity O(n m). There are 2
senarios:
\subsection{$n \leq m^2$}
\subsubsection{Algorithm}
In this senario, standard closest pair algorithm can meet the requirement.
\subsubsection{Time complexity}
\begin{align*}
T_{close}(n) =& O (n \log n) < a n \log n < a n \sqrt{n} \leq a n m\\
T_{close}(n) =& O(nm)
\end{align*}
\subsection{$n > m^2$}
The sqaure of dimension (m, m) can be divided into $m^2$ unit squares with
dimension (1, 1). Because $d(S_1, S_2) \geq 1$, each square can contain at most
4 points. As there are more points(n) than sqaures($m^2$), one square has more
than 1 points, and the closest pair must be in the same square or adjacent squares.
\subsubsection{Algorithm}
\begin{lstlisting}
(Point, Point) findClosestPair(ArrayList<Point> points) {
	closestDistance = 1000m;
	Point p1 = NULL;
	Point p2 = NULL;
	for (Point point1: points) {
		for (Point point2: point1.findAllPointsInAdjacentSqaures() {
			if (distance(point1, point2) < closestDistance) {
				p1 = point1;
				p2 = point2;
				closestDistance = distance(point1, point2);
			}
		}
	}
	return (p1, p2);
}
\end{lstlisting}
\subsubsection{Time complexity}
The outer for loop has $n$ iterations for all the points and the inner for loop
has at most $8 \times 9 = 72$ iterations (4 points per sqaure, 9 adjacent
squares, at maximum). All the processing within the inner loop and function
findAllPointsInAdjacentSqaures() cost time O(1). So the total time
complexity(including the preprocessing) is
\begin{align*}
T_{close}(n) = n 72 O(1) + O(nm) = O(nm)
\end{align*}
\end{document}
