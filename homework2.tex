\documentclass{article}
\usepackage{listings}
\usepackage{amsmath}
\usepackage{fullpage}
\usepackage{tabularx}
\usepackage{graphicx}
\usepackage{cite}
\begin{document}
\lstset{language=Java}
\title{CS350 homework 2}
\author{Yuchen Hou}
\maketitle

\section{Partition}
\begin{lstlisting}
Integer partition(ArrayList<Integer> A, Integer p, Integer q) {
	if p >= q
		return q;
	Integer pivot = A[p]
	Integer index = p;
	while p < q {
		while A[p] < pivot
			p++;
		while A[q] > pivot
			q--;
		if p < q
			swap(A[p], A[q]);
	}
	swap (A[index], A[q]);
	return q;
}
\end{lstlisting}

\section{Insertion sort}
This proof is independent of the initial ordering of array A, e.g. whether A is
monotonically decreasing or not. In fact, the time complexity of inserting one
number into an array with n-a element is always $O(n^2)$.

\subsection{Recursive relation}
$T_{insert}(n) = T_{insert}(n - 1) + O(n)$

\subsection{Statement}
$T_{insert}(n) = O(n^2)$

\subsection{Proof by induction}
Induction hypothesis: $\forall m < n, T_{insert}(m) = O(m^2) < b m^2$
\begin{align*}
T_{insert}(n)
&= O((n-1)^2) + O(n)\\
&< b(n-1)^2 + an\\
&= b n^2 + (a - 2b)n + b\\
&< b n^2 \text{ // choose a large enough b}
\end{align*}

\section{Iqsort}
$T_{iq}(n) = T_{partition}(n) + T_{quick}(r-1) + T_{insert}(n-r)$

\subsection{Best case: A is already sorted}
\begin{align*}
T_{quick}(r-1) &= O((r-1) \log (r-1))\\
T_{insert}(n-r) &= O(n-r)\\
T_{partition}(n) &= O(n)\\
T_{iq}(n)
&= \min_r (T_{partition}(n) + T_{quick}(r-1) + T_{insert}(n-r))\\
&= \min_r(cn + a(r-1) \log (r-1) + b(n-r))\\
&= ((c+b)n -b) \text{ // choose r = 1}\\
&< dn \text{ // choose a large enough d}\\
\implies T_{iq} &= O(n)
\end{align*}

\subsection{Average case}
\begin{align*}
T_{quick}(r-1) &= O((r-1) \log (r-1))\\
T_{insert}(n-r) &= O((n-r)^2)\\
T_{partition}(n) &= O(n)\\
T_{iq}(n)
&= \sum_r \frac{1}{n}(T_{partition}(n) + T_{quick}(r-1) + T_{insert}(n-r))\\
&= \sum_r \frac{1}{n}(cn + a (r-1) \log (r-1) + b(n-r)^2)\\
&= \int_1^{n+1} \frac{1}{n}(cn + a (r-1) \log (r-1) + b(n-r)^2) d r\\
&< \frac{a}{2} n \log_2(n) - a n/(4\log_e(2)) + cn + \frac{b}{3} n^2 \\
&< dn^2 \text{ // choose a large enough d}\\
\implies T_{iq} &= O(n^2)
\end{align*}

\subsection{Worst case: A is in reverse order}
\begin{align*}
T_{quick}(r-1) &= O((r-1)^2)\\
T_{insert}(n-r) &= O((n-r)^2)\\
T_{partition}(n) &= O(n)\\
T_{iq}(n)
&= \max_r (T_{partition}(n) + T_{quick}(r-1) + T_{insert}(n-r))\\
&= \max_r(cn + a(r-1)^2 + b((n-r)^2)\\
&= \max_r (cn + bn^2 -2bnr -2ar + (a+b)r^2)\\
&< dn^2 \text{ // choose a large enough d}\\
\implies T_{iq} &= O(n^2)
\end{align*}

\section{Mixsort}

\subsection{Best case: A is already sorted}

\subsubsection{Recursive relation}
\begin{align*}
T_{mix}(n) &= T_{partition}(n) + T_{mix}(r-1) + T_{insert}(n-r)\\
T_{partition}(n) &= O(n)\\
T_{insert}(n-r) &= O(n-r)
\end{align*}

\subsubsection{Statement}
$T_{mix}(n) = O(n)$

\subsubsection{Proof by induction}
Induction hypothesis: $\forall m < n, T_{mix}(m) = O(m ) < am$
\begin{align*}
T_{mix}(n)
&= \min_r (T_{partition}(n) + T_{mix}(r-1) + T_{insert}(n-r))\\
&< \min_r (cn + a(r-1) + b(n-r))\\
&< cn + b(n-1) \text{ // choose a very large a, r = 1}\\
&< (c + b)n -b\\
&< an \text{ // choose a large enough a}\\
\implies T_{mix}(n) &= O(n)
\end{align*}

\subsection{Average case}

\subsubsection{Recursive relation}
\begin{align*}
T_{mix}(n) &= T_{partition}(n) + T_{mix}(r-1) + T_{insert}(n-r)\\
T_{partition}(n) &= O(n)\\
T_{insert}(n-r) &= O((n-r)^2)
\end{align*}

\subsubsection{Statement}
$T_{mix}(n) = O(n^2)$

\subsubsection{Proof by induction}
Induction hypothesis: $\forall m < n, T_{mix}(m) = O(m^2) < am^2$
\begin{align*}
T_{mix}(n)
&= \sum_r \frac{1}{n}(T_{partition}(n) + T_{mix}(r-1) + T_{insert}(n-r))\\
&< \sum_r \frac{1}{n}(cn + a(r-1)^2 + b(n-r)^2)\\
&< \int_1^{n+1} \frac{1}{n}(cn + a(r-1)^2 + b(n-r)^2) dr\\
&< \frac{a}{3} n^2 + cn + \frac{b}{3} n^2 \\
&< an^2 \text{ // choose a large enough a}\\
\implies T_{mix} &= O(n^2)
\end{align*}

\subsection{Worst case: A is in reverse order}

\subsubsection{Recursive relation}
\begin{align*}
T_{mix}(n) &= T_{partition}(n) + T_{mix}(r-1) + T_{insert}(n-r)\\
T_{partition}(n) &= O(n)\\
T_{insert}(n-r) &= O((n-r)^2)
\end{align*}

\subsubsection{Statement}
$T_{mix}(n) = O(n^2)$

\subsubsection{Proof by induction}
Induction hypothesis: $\forall m < n, T_{mix}(m) = O(m) < am^2$
\begin{align*}
T_{mix}(n)
&= \max_r (T_{partition}(n) + T_{mix}(r-1) + T_{insert}(n-r))\\
&< \max_r(cn + a(r-1)^2 + b(n-r)^2)\\
&= \max (cn + b(n-1)^2, cn + a(n-1)^2)\\
&= an^2 + (c-2a)n + a \text{ // choose a very large a}\\
&< an^2 \text{ // choose a large enough a}\\
\implies T_{mix} &= O(n^2)
\end{align*}

\end{document}
