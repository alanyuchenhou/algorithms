\documentclass{article}
\usepackage{listings}
\usepackage{amsmath}
\usepackage{fullpage}
\usepackage{tabularx}
\usepackage{graphicx}
\usepackage{cite}
\begin{document}
\lstset{language=Java, tabsize=4}
\title{CS350 homework 9}
\author{Yuchen Hou}
\maketitle

\section{NP reduction}
We have the following observations:
\begin{align*}
A \leq_m B \implies \exists T_1 \land \forall x_1 A(x_1) == yes \iff B(T_1(x_1))
== yes \\
B \leq_m C \implies \exists T_2 \land \forall x_2 B(x_2) == yes \iff C(T_2(x_2))
== yes
\end{align*}
where $T_i$ is a deterministic polinomial time algorithm. Now we construct
another deterministic polinomial time algorithm: $T_3(x_1) = T_2(T_1(x_1))$. We
have:
\begin{align*}
&\forall x_1 A(x_1) == yes \iff B(T_1(x_1)) == yes \iff C(T_2(T_1(x_1))) == yes
\iff C(T_3(x_1)) == yes \\
&\implies A \leq_m C
\end{align*}

\section{NP solution}
\begin{lstlisting}
Boolean Q(G) {
	guess(path); // |path| < p(|G|)
	if (check(G, path) == yes) // complexity < p(|G|)
		return yes;
	else
		crash;
}
\end{lstlisting}
Note:
\begin{enumerate}
  \item guess(path): returns a path on G such that every node in G is covered
  exactely once
  \item check(G, path): returns yes if path is on G and path covers every node
  in G exactly once
  \item $|G| = |V| + |E|$
\end{enumerate}

\section{NP solution}
\begin{lstlisting}
Boolean Q(G) {
	guess(path); // |path| < p(|G|)
	if (check(G, path) == yes) // complexity < p(|G|)
		return yes;
	else
		crash;
}
\end{lstlisting}
Note:
\begin{enumerate}
  \item guess(path): returns a path on G such that every node in G is covered
  \item check(G, path): returns yes if path is on G and path covers every node
  in G
\end{enumerate}

\section{SAT}
\begin{lstlisting}
Boolean isEquivalent(C1, C2) {
	C3 = makeXnor(C1, C2); // |C3| < p(|C1|+|C2|)
	if (isSatisfiable(C3)) { // complexity < p(|C1|+|C2|)
		return false;
	} else {
		return true;
	}
}
\end{lstlisting}
Note: makeXnor function creates a new circuit C3, which feeds its input $x = (x_1,
\ldots x_n)$ to the input of C1 and C2, then feeds the output of C1 and C2 to a
xnor gate X, and feeds the output of X to the output of C3. Now C1 and
C2 are equivalent $\iff \forall x C1(x) == C2(x) \iff \forall x C3(x) == flase
\iff !isSatisfiable(C3)$.

\end{document}
