\documentclass{article}
\usepackage{listings}
\usepackage{amsmath}
\usepackage{fullpage}
\usepackage{tabularx}
\usepackage{graphicx}
\usepackage{cite}
\begin{document}
\lstset{language=Java, tabsize=4}
\title{CS350 homework 8}
\author{Yuchen Hou}
\maketitle

\section{RSA}
Steps:
\begin{enumerate}
  \item fractor n into p and q: p = 43481; q = 242399;
  \item calculate private key = (d = 385818449, n = 10539750919);
  \item encode cipher text to cipher code: 4739112828 4663943684 5483262135
  \item decrypt cipher code to plain code: 2883783099 456562021 9711147868
  \item decode plain code to plain text: exxgtxg0!! iorore0ea !taaaitxrx
\end{enumerate}

\section{n-ary Huffman coding}
Assume the algorithm uses n digits to encode a character set.
\begin{lstlisting}
Class Node {
	public Character character;
	public Float probability;
}
Tree<Node> huffman(Set<Node> characters) {
	PriorityQueue<Node> queue = new PriorityQueue<Node>();
	Tree<Node> tree = new Tree<Node>();
	for (Node character: characters) {
		queue.add(character); // lowest probability -> highest priority.
		tree.addNode(character);
	}
	while (queue.size() != 0) {
		Node parent = new Node("", 0);
		tree.addNode(parent);
		for (Integer i: range(0, n)) {
			Node child = queue.remove();
			tree.getNode(parent).probability += child.probability;
			tree.addEdge(parent, child, i);
		}
		queue.add(parent);
	}
	return tree;
}
\end{lstlisting}
The code of each character is the digit sequence on the path from the root to
its corresponding leaf node.

\section{Security protocols}
A few current techniques are as follows:
\begin{enumerate}
  \item Key agreement or establishment
  \item Entity authentication
  \item Symmetric encryption and message authentication material construction
  \item Secured application-level data transport
  \item Non-repudiation methods
  \item Secret sharing methods
  \item Secure multi-party computation
\end{enumerate}
\end{document}
